\chapter{Einleitung}
\label{ch:intro}

\citet{Brockhoff2006} formuliert ein Ergebnis, dass in vielen Banken heterogene, unflexible und eher zufällig gewachsene Architekturen dominieren. Das Ergebnis leitet sich aus dem historischen Kontext ab. Banken sind frühe Anwender von IT-Systemen, woraus sich über die Jahre eine komplexe IT-Architektur etablierte.

Das Ergebnis aus der veralteten und paralysierten IT-Architektur von Banken lässt sich mit der Bezeichnung Legacy-IT \cite{Dorschel2018, Strietzel2018} zusammenfassen.
Innovationsdruck stößt auf Veränderungswiderstand und einer der effektivsten Innovationen der letzen Jahre, das Cloud-Computing hinterfragt die gesamte Architektur von Organisationen mit Legacy-IT.
\medskip
\\
Die Anpassung von bewährten Standardlösungen auf eine problematische Umgebung \cite{Bussmann2006, Brockhoff2006, Strietzel2018} lindert oftmals nur die Symptome und ist, gerade in den Systemen der Legacy-IT mit hohem Aufwand verbunden. Ein grundlegendes Re-Design nach \citet[27]{Bussmann2006} ist durch Cloud-Technologie sehr greifbar.
\medskip
\\
Hierfür entstehen weitere Einschränkungen, die aus den externen Rahmenbedingungen rühren. Als eine Einschränkung könnten die regulatorischen Anforderungen \cite{MaRisk:2017}, \cite{BAIT:2018}, \cite{eba:2019} angesehen werden \cite{recht/Bornemann2018}. Nach der Betrachtung von diesen kann festgestellt werden, dass aus Sicht eines Kreditinstiuts eine Wechselwirkung zwischen Veränderungsdruck und Widerstand herrscht. Eine wichtige Frage ist daher, worin die Ursachen für eine mangelnde Durchsetzungsfähigkeit für Veränderungen liegen.
\medskip
\\
In der IT-Architektur der Institute entstehen Probleme\footnote{vgl. \cite{mci/Disterer2011}} und einschränkende Faktoren, die nicht nur Auswirkungen auf den Betrieb, sondern auch auf die Entwicklung der IT-Systeme haben. Für die Entwicklung und Betrieb von IT-Systemen sind in jedem Institut Regelprozesse zu definieren \cite{MaRisk:2017}. 
\medskip
\\
Zum Beispiel müssen Standardanwendungen von Entwicklern vor ihrem Einkauf nicht nur ausreichend getestet, sondern auch auf die jeweiligen internen Prozesse angepasst werden. Im Ansatz \ac{SaaS} ist die Auslagerung von IT-Ressourcen eine gängige Praxis zum Betreiben von Anwendungen. Dadurch werden zusätzliche Schritte im Risikomanagement der Institute angefordert \cite{MaRisk:2017}.
\medskip
\\
Gängige Standards der Softwareentwicklung befinden sich mit neuen Technologien, Ansätzen und Anwendungen im ständigen Wandel. Vor ihrem geplanten Einsatz müssen mögliche Auswirkungen auf die Kontrollverfahren der Institute analysiert werden \cite{MaRisk:2017}.
Die IT-Architektur ist, bedingt durch Impulse aus neuen Technologien \cite{Bussmann2006}, ständigem Innovationsdruck ausgesetzt und muss sich \enquote{kontinuierlich justieren} \cite{Bussmann2006} können. 
\medskip
\\
Die Kredit- und Finanzdienstleistungsinstitute in Deutschland können sich Veränderungen nicht mehr entziehen. Insbesondere ist das aufgrund der Disruptivität von einigen Technologien \cite{Fernandez:2020} und Ansätzen bedingt. Eine bevorstehende digitale Transformation von Banken und Finanzdienstleistern bietet für anpassungsfähige Institute einzigartige Möglichkeiten sich neu zu etablieren \cite{Gupta:2017}. Gleichzeitig besteht auch bei Nichtbeachtung von Technologietrends die Gefahr einer Verdrängung von kleineren und effizienten Betrieben, die schnell auf die Impulse reagieren.
\medskip
\\
Lloyd Blankfein verkündete, als Vorsitzender und \ac{CEO} von Goldman Sachs \emph{\enquote{Wir sind ein Technologieunternehmen. Wir sind eine Plattform.}} \cite{Gupta:2017}. Tatsächlich könnte aus Sicht der Digitalen Transformation eine Bank als Plattform für Finanzdienstleistungen definiert werden und neue Wertschöpfungsketten erzeugen. 
Bevor die Opportunitäten in Angriff genommen werden können stellt sich die Frage welche digitalen Anlagen priorisiert werden sollen.
\medskip
\\
%
% Section: Motivation
%
\section{Motivation}
\label{sec:intro:motivation}

Die nachfolgende Arbeit ist aus der Frage entstanden, wie man die die CI Plattform Jenkins in einer Bank skalierbarer und flexibler gestalten könnte.

Anfänglich wurde der Fokus auf CI/CD gerichtet und anschließend Themen wie Cloud-Computing, Container-Orchestrierung und Microservices vorgenommen. Dabei wurde sich mit der Entwicklung von Docker-Images für Jenkins Agenten\footnote{Slave innerhalb einer Master-Slave Konfiguration in Jenkins \cite{Google:GKEJenkins, Pathania2017}} der \ac{CI} Plattform Jenkins beschäftigt. Ziel war hierbei eine flexibel einsetzbare Umgebung zum Ausführen von verteilten Builds\footnote{In diesem Zusammenhang die Erstellung von Software mit dem Build-Tool Maven}.
\medskip
\\
Ansätze wie Cloud-Computing, \ac{IaaS}, \ac{SaaS}, DevOps und Agilität werden zum Standard für die Softwareentwicklung. Banken benötigen für die Umsetzung von neuen Technologien und Methoden umfangreiche Anpassungen ihrer Prozesse. Selbst eingekaufte Standardanwendungen unterliegen nicht selten umfangreichen Analysen und Anpassungen mit hohen Kosten. 
\medskip
\\
Es sollte sich bezüglich der internen Kontrollverfahren und IT-Strategie die Frage gestellt werden, wie mit ständigen Veränderungen in der IT umzugehen ist. Ein angemessenes Risikomanagement sollte Veränderungen fördern. Eine mangelnde Anpassungsfähigkeit und Geschwindigkeit stellt möglicherweise ein höheres Risiko dar.
\medskip
\\
Agile Prinzipien und Best-Practices müssen durch diese umfangreichen Anpassungen gegenüber der Legacy-IT klein bei geben. Die Legacy-IT trotzt der Innovation. Daher hat ein Innovator der Legacy-IT zu trotzen.
%
% Section: Ziele
%
\section{Ziel der Arbeit}
\label{sec:intro:goal}

Die nachfolgende Arbeit soll sich mit den Hintergründen der bereits länger bekannten Einschränkungen der IT-Architektur von Banken \cite{Brockhoff2006, Bussmann2006, Gupta:2017, Strietzel2018, Dorschel2018, Manz2018} beschäftigen.
Diese Einschränkungen haben Auswirkungen auf den Betrieb für agile Softwareentwicklung in der IT von Banken und es stellt sich die Frage:
\begin{quote}
    \enquote{Welche Technologien und Ansätze eignen sich für eine skalierbare und flexible Softwareentwicklungsumgebung in Banken?}
\end{quote}
%
Die Einführung von neuen Standards\footnote{Cloud-Computing, Microservices, DevOps, CI/CD, Scrum, Agilität}
für die Umsetzung von Veränderungsprozessen erfolgt im Finanzwesen jedoch eingeschränkt und verspätet. Für ihre Einführung unterliegen diese Ansätze umfangreichen Anpassungen, die ihre Vorteile negieren oder werden durch neue Technologien und Trends kurz nach ihrer verspäteten Einführung überflüssig. Die in \cite{Brockhoff2006} geforderte \ac{SOA} wurde durch Microservices ersetzt \cite[S. 80f]{Dorschel2018}.
\medskip
\\
Daher wird als Voraussetzung für die Lösung dieser Forschungsfrage bedingt, Faktoren im Innovationsverlauf \cite{Alt2017} für eine bessere Transformationsfähigkeit \cite{Koch2016} der Legacy-IT in Banken zu identifizieren. Eine zweite Forschungsfrage:
\begin{quote}
    \enquote{Welche einschränkenden und antreibenden Faktoren gibt es für die IT von Banken, um Veränderungen durchzuführen?}
\end{quote}
Daher hat die Aufgabe einen investigativen Charakter bezüglich der Problemursachen. Als Methodik wird sich am Design-Thinking Prozess orientiert \ref{def-design-thinking}.

Hierzu werden die Rahmenbedingungen für Veränderungen aus verschiedenen Perspektiven durchleuchtet:
\begin{itemize}
    \item \emph{Finanzwesen}, anhand einer Fallstudie \cite{Gupta:2017} über Goldman Sachs' Transformation
    \item \emph{Technologietrends}, die Impulse geben und später Disruptivität oder Diskontinuität \cite{Fernandez:2020} erzeugen könnten
    \item \emph{Bankaufsicht}, die mit ihren Vorgaben \cite{MaRisk:2017}, \cite{BAIT:2018} einen Rahmen für das Finanzwesen festlegt, mit Auswirkung auf die IT
\end{itemize}
Aus dem genannten Zusammenhang werden die gegenseitigen Einflüsse identifiziert für einschränkende und antreibende Faktoren der Veränderungsprozesse analysiert. 
\medskip
\\
\emph{Veränderungen} sind aufgrund der Auswirkungen von Technologietrends unvermeidbar und müssen mittlerweile kontinuierlich erfolgen \cite{Bussmann2006, Alt2017, Fernandez:2020}. Daher stellt sich eine Frage worin die Lösung liegen könnte: 
\begin{quote}
    \enquote{Wie sollten Banken bezüglich ihrer IT mit kontinuierlichen Veränderungen umgehen?}
\end{quote}
Die Arbeit beschäftigt sich daher mit einem Paradigmawechsel der regulierten Finanz-IT, um einen Impuls zu geben die nötigen Veränderungsprozesse effizient und kontinuierlich durchsetzen zu können.


%
% Section: Struktur der Arbeit
%
\section{Aufbau der Arbeit}
\label{sec:intro:structure}

\paragraph{Aufbau der Methodik}
Das Problem wird zuerst aus der Sicht der \ac{SEU} definiert und daraus die allgemeinen Ursachen verfolgt. Von den Hintergründen aus werden unterschiedliche Perspektiven durchleuchtet und wesentliche Punkte deduktiv ausgearbeitet und bewertet.

Die Arbeit verwendet Schritte aus dem \emph{Design-Thinking} Prozess (Kap. \ref{def-design-thinking}). Daher werden Kap. \ref{ch:background} schon induktiv einfache Ansätze für erste Lösungen ad hoc vorgeschlagen und beschrieben. Dies bezweckt eine frühe Datenerhebung mit Gestaltungsprozessen, sodass Ansätze und Anforderungen von der angenommenen Perspektive heraus vorgeschlagen werden können. Diese können sich somit gegenseitig im Analyseteil ausschließen, zur Filterung vor der Bewertung in Kap. \ref{chapter:evaluation}. In diese fließen anschließend die festgehaltenen Ergebnisse.
\bigskip
\\
Die Arbeit ist unter drei wesentlichen Bereichen gegliedert. 
Im Bereich Grundlagen (Kap. \ref{chapter:grundlagen}) werden bekannte Probleme und Ansätze aus verwandten Werken zusammengefasst und der aktuellen Stand der Technik dargestellt und daraus erste Ansätze mit gängigen Standards vorgeschlagen. 
\medskip
\\
Im nächsten Bereich (Kap. \ref{ch:background} findet die Untersuchung zu den Rahmenbedingungen der Softwareentwicklungsumgebung in Banken statt. Hierin werden die einschränkenden Faktoren ermittelt. Zusätzlich werden ihre Auswirkungen und Risiken für das Geschäftsmodell untersucht. Aus der Literaturrecherche werden mögliche Ansätze für eine angemessene Umgebung für die Softwareentwicklung mit aktuellen Technologien und Methoden abgeleitet und erste Lösungsansätze vorgeschlagen. Hierbei wird auf die Regulatorik eingegangen. Im letzten Abschnitt findet eine Synthese der Ergebnisse\footnote{aus dem Design-Thinking Prozess in (Kap. \ref{def-design-thinking})} statt, die anschließend analysiert und nach jeder Iteration kontinuierlich verbessert werden.
\medskip
\\
Im anschließenden Bereich (Kap. \ref{chapter:evaluation}) findet die Bewertung und Zusammenfassung der Ergbenisse statt. Zuerst wird die Methodik und der Weg zur Lösung der Forschungsfragen rückblickend betrachtet und bewertet. Als nächstes werden die wesentlichen Ergebnisse aus dem vorigen Bereich bewertet, festgehalten und zusammengefasst.
