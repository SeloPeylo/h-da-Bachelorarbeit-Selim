\chapter{Grundlagen}

\section{Verwandte Werke}
Bussman et al. erklärt in \cite{Bussmann2006} Impulse, die durch neue Technologien ausgelöst werden können. Dabei wird auch auf die damit verbundene mögliche Modernisierung von Anwendungen im Finanzdienstleistungsbereich eingegangen. Diese wird durch einen Ansteig der Anforderungen in produktionsnahen, operativen Systemen \cite{Bussmann2006, Brockhoff2006} bedingt. Hierfür nennt er zwei spezifische Ansätze. Danach wird die Rolle der IT-Architektur in Verbindung mit der IT-Strategie beschrieben und eine hohe Flexibilität und Schnelligkeit gefordert. Für eine kontinuierliche Anpassung der IT-Architektur, die für den Erfolg wichtig ist \cite{Bussmann2006}, werden die damit verbundenen Schwierigkeiten genannt. Anschließend werden in einem Ausblick große Auswirkungen der IT auf die etablierten Geschäftsmodelle erwartet. Die Ergebnisse von Bussman et al. sind sehr relevant, da die Auswirkungen von neuen Technologien auf das Geschäftsmodell vorausgesehen werden und Bedingungen für den Umgang mit Technologientrends gestellt werden.
\medskip
\\
Brockhoff beschreibt in \cite{Brockhoff2006} die Hintergründe und Probleme der IT-Architektur in Banken und stellt Anforderungen an ihre IT-Architektur. Dabei fordert er die eine \ac{SOA}, um die mangelnde Flexibilität und Skalierbarkeit der alten Architektur zu beseitigen. Die Schilderungen von Brockhoff geben die Probleme in der unflexiblen IT-Architektur aus Sicht des Finanzwesens wieder.
\medskip
\\
Disterer beschreibt in \cite{mci/Disterer2011} eine problematische Inbetriebnahme von Anwendungen durch die mangelnde Beachtung von nichtfunktionalen Anforderungen bei der Entwicklung. Hierzu stellt er als Lösungsansatz eine Erweiterung eines ITILv3 Prozesses mit Quality Gates, die als Synchronisationspunkt für die Koordination zwischen Entwicklung und Betrieb dienen. Der Ansatz von Disterer gibt eine Referenz für eine bessere Koordination zwischen Entwicklung und Betrieb.
\medskip
\\
Alt et al. erklärt bezogen auf das IT-Management in \cite{Alt2017} das Potenzial von DevOps zur Steigerung der Innovationsfähigkeit. Dieser ergibt sich aus Kontinuitätskonzepten und ihrer Verbindung mit einer inoovationsfokussierten IT-Strategie. Hierfür erklärt er die DevOps Prinzipien, worin er für das Kontinuitätsprinzip auf den Ansatz \emph{Continuous Delivery} eingeht. 
\medskip
\\
Pathania zeigt in \cite{Pathania2017} wie Jenkins mit Kubernetes, Docker und Public-Cloud skaliert werden kann. Dabei konfiguriert er Jenkins für die Nutzung innerhalb einer Public-Cloud und Kubernetes. Er veranschaulicht einen Ansatz, um den Jenkins Master über Replikas zu skalieren. Zudem beschreibt er die Skalierung von Jenkins Agenten für die Verteilung der Builds. Für die Agenten werden Images erzeugt, woraus Jenkins mithilfe eines Docker Plugins einen Container für den Build generiert. Für die Agenten erstellt er ein Image aus Ubuntu und installiert die Build-Tools, wie in jedem anderen Ubuntu System.
Dadurch erzeugt er die Images über einen Container, welches vorher konfiguriert wurde. Insbesondere beschreibt Pathania ausführlich die einzelnen Schritte, die für die Orchestrierung der ganzen Anwendung mit Kubernetes erforderlich sind.

\section{Stand der Technik}

Zuletzt hat sich die IT in Betrieben vor allem durch einige Schlüsseltechnologien verändert. Für die IT-Architektur hat die Cloud-Computing Technologie zusammen mit Public Cloud Anbietern einen entscheidenden Impuls gegeben. Anwendungen können in kürzester Zeit in großem Umfang skaliert werden. Infrastruktur und Software wird als Dienstleistung bereitgestellt. Kurze Entwicklungszyklen durch agile Methoden sorgen für eine kontinuierliche Weiterentwicklung der Anwendungen, die widerrum dank DevOps und Continuous Deployment und -Delivery in kurzer Zeit bereitgestellt werden.
\medskip
\\
Software hat eine Schlüsselrolle für digitale Innovation in allen Bereichen \cite{Alt2017}, die wichtiger denn je geworden ist. So entstehen gegenüber etablierten Unternehmen mit properitären Infrastrukturen \cite{Bussmann2006}, Technologieunternehmen mit effizienten Geschäftsmodellen. Sie integrieren mit unterstützenden Anwendungen IT und Geschäftsmodell \cite{Bussmann2006}. 
\medskip
\\
Die IT ist ständigen Veränderungen ausgesetzt. Technologietrends erzeugen immer wieder neue Impulse und Betriebe mit properietären Infrastrukturen müssen sich neu ausrichten \cite{Bussmann2006}. Im Finanzwesen entstehen aus Technologie-Start-Ups Banken mit direktem Vertrieb über das Internet, wie zum Beispiel N26. Dazu entstehen Banken, die Schnittstellen für Unternehmen anbieten und somit eine reine Plattform für Finanzdienstleistungen darstellen, beispielsweise die Solarisbank. Auch etablierte Banken haben veränderte Rahmenbedingungen und Impulse erkannt und passen sich dem aktuellen Stand der Technik an \footnote{vgl. \cite{Gupta:2017}}.

\subsection{Fallstudie über Goldman Sachs}
Goldman Sachs sieht sich nicht als Finanzdienstleister, sondern als ein Technologieunternehmen und als eine Plattform \cite{Gupta:2017}. Gupta und Simonds beschreiben den Weg der digitalen Transformation, den Goldman Sachs eingegangen ist: \footnote{In: Goldman Sachs' Digital Journey \cite{Gupta:2017}}

Goldman Sachs hat 2013 durch ihre Technologieplattform namens Marquee ihre hauseigenen Daten, Analysen und Risikomanagementanwendungen an ihre Geschäftskunden verfügbar gemacht. Die Kunden konnten dadurch hochwertigen Daten in ihre eigenen Systeme integrieren. Dabei stellt sich die Frage, ob es kein Wettbewertsnachteil ist die Daten anzubieten.

Eine Welle von regulatorischen Veränderungen wurde durch die Auswirkungen der Finanzkrise 2008 herbeigeführt, um zukünftige systemische Krisen zu vermeiden. Auf internationaler Ebene wurden umfangreiche Reformen wurden durch Basel III herbeigeführt. Um finanzielle Unterstützung von der U.S. Regierung zu erhalten hat Goldman Sachs sich strukturell von einer Investmentbank zu einer \enquote{bank holding company} verwandelt. Dadurch wurde das Unternehmen mit zusätzlichen Kosten und Regulierungen konfrontiert.

Technologie hat im Finanzwesen und in Goldman Sachs schon immer eine wichtige Rolle gespielt um sie transparenter und besser im Umgang mit Risiken zu machen. Die Analyse von großen Datenmengen zum Mitigieren von Risiken in Echtzeit ist zur Norm geworden. Von Algorithmen erzeugte Metriken sind gegenüber den Instinkten der Institute wertvoller geworden.

Die Führungskräfte von Goldman Sachs arbeiteten an der Wertschöpfung dieser Technologien. Unter anderem wurde eine \emph{hybride Cloud}\footnote{Gemeinsamer Einsatz einer extern und einer intern betriebenen Cloud-Plattform} benutzt, um einen so genannten \enquote{Data Lake} zu erzeugen. Methoden aus Mashine-Learning wurden verwendet, um daraus Erkenntnisse zu generieren. 

Die Geschwindigkeit und Auswirkung der Technologien wurde nach der Finanzkrise nur noch beschleunigt. Die Zeit und Kosten zur Erzeugung von neuen Technologien hat sich durch die Ausbreitung von Cloud Plattformen, Open-Source Software und Schnittstellen erheblich reduziert. Dieser Paradigmenwechsel hat Goldman Sachs' Fähigkeit zur Entwicklung und Integration von Technologien verstärkt. Parallel dazu wurde auch ihr Wettbewerb verstärkt durch die Entstehung von fintechs\footnote{Zusammensetzung aus \enquote{finance} und \enquote{technology}. Bezeichnet in der Regel Technologien und Start-Ups im Finanzwesen}.

Darren Cohen\footnote{Vorsitzender von Goldman's PSI Group, vgl. \cite{Gupta:2017}} äußert sich zu FinTechs, dass von ihnen keine direkte Gefahr im Bereich Wertpapierhandel für Goldman Sachs ausgeht. Die Abwicklung ist hoch reguliert und erfolgreiche Unternehmen müssen eine hohe Bilanzsumme, hohe Infrastrukturinvestitionen und einen erheblichen, globalen Einfluss aufweisen. Diese Anforderungen sind schwierig für neue Marktteilnehmer und für junge FinTechs unüberwindlich. 

Dazu schließt sich Chavez\footnote{\ac{CFO}, ehemaliger \ac{CIO} von Goldman Sachs, vgl. \cite{Gupta:2017}} an: \enquote{Die Fintech-Industrie ist super interessant und innovativ, und wir beteiligen uns als Investoren und Kunden. Es besteht jedoch nach wie vor ein kritischer Bedarf an regulierten Banken.} \cite{Chavez:2017}.

Die Integration von neuen Technologien war für Goldman Sachs eine Opportunität und unvermeidbar zugleich, die weitreichende Auswirkungen für ihre Zukunft beinhaltet.
Der Vorstand von Goldman Sachs hat den Bedarf der Kostensenkung durch einen effizienteren Betrieb, die Unterstützung des Kerngeschäfts und die Wertschöpfung von neuen Geschäftsmodellen erkannt.

Unterstützt vom Vorstand, hat Chavez und sein Team eine IT-Strategie ausgearbeitet, die sich auf Effizienzsteigerung im Unternehmen konzentriert hat. Darin wurde die Entwicklung von internen und externen Plattformen vorgesehen. Die Automatisierung von Geschäftsprozessen wurde durch die Zentralisierung von Kernelementen vorangetrieben, wodurch Redundanzen entfernt wurden.

Die unterschiedlichen Geschäfte wurden bisher in \enquote{Silos} als unabhängige und eigenverantwortliche Einheiten ausgeführt. Es stellte sich die Frage, was ihre gemeinsamen Nenner sind. Ein Ansatz war in einer gemeinsamen technologischen Plattform zu arbeiten und in kleineren Teams sich darauf zu konzentrieren, was das jeweilige Geschäft ausmacht. Zu ihrer Unterstützung unterliegt ihnen ein größeres gemeinsamen Team.

Dieser Ansatz geht jedoch mit vielen Schwierigkeiten einher und erfordert die Zusammenarbeit aller Beteiligten. Die Geschäftseinheiten müssen für die Entwicklung von gemeinsamen Plattformen kurzfristige Kompromisse eingehen und als finanziell eigenverantwortliche Einheiten die Kosten tragen. Es gibt einserseits eine gemeinsame Vision, der Fortschritt besteht jedoch darin die historischen, vertikalen Barrieren zu durchbrechen und in einer digitalen Welt zu arbeiten.

\cite{Gupta:2017}


\subsection{Methoden und Prozesse der Softwareentwicklung}

\paragraph{Continuous Integration}

\paragraph{DevOps}
Alt et al. erklärt die Hintergründe von DevOps:

DevOps ist eine Bezeichnung, die durch die Zusammensetzung der Wörter \enquote{Development} und \enquote{Operations} stammt und entstand aus den \emph{Problemen bei der Inbetriebnahme von Software \cite{mci/Disterer2011}}.
Aus einer schnellen, flexiblen und nutzerzentrierten Umsetzung von funktionalen Anforderungen resultieren Benutzerakzeptanz und Kundenzufriedenheit. 

Techniken, Prozesse und Tools von DevOps zielen darauf ab typische Probleme in der Zusammenarbeit zwischen Entwicklung Betrieb entgegen zu wirken. In Konsequenz soll die Kundenzufriedenheit verbessert werden. Die Automatisierung von Routineaufgaben und Wandel in der Zusammenarbeitskultur hat darin eine zentrale Bedeutung. Für die DevOps-Prinzipien existieren keine standardisierten Vorgehensbeschreibungen. Die Einführung ist sehr individuell und Unternehmen passen die Prinzipien an ihre eigene Situation an. 

DevOps steht für \enquote{Culture}, \enquote{Automation}, \enquote{Measurement} und \enquote{Sharing} \cite{humble:2011}. 
\begin{itemize}
    \item \emph{Kulturwandel} zu einer gemeinsamen Verantwortung aller Beteiligten für die Auslieferung von Qualitätssoftware
    \item \emph{Automatisierung} der Prozesse von Entwicklung, Test und Bereitstellung bis hin zur Produktivnahme als Schlüssel für kürzere Durchlaufzeiten, Fehlervermeidung und schnelleres Feedback
    \item \emph{Kennzahlen}, die miteinander verknüpft als Beitrag zur geschäftlichen Wertschöpfung und Produktivitätssteigerung durch faktenbasierte Einstufung der aktuellen Leistungen und Definition von überprüfbaren Verbesserungszielen
    \item \emph{Teilen} zum Beispiel von Wissen, Tools, Infrastruktur und die Würdigung von Erfolgen als Prinzip um die Zusammenarbeit zwischen den Beteiligten zu prägen
\end{itemize}

\cite{Alt2017}

\paragraph{Continuous Delivery}

\paragraph{ITIL}
Alt et al. beschreibt ITIL im Rahmen des Innovationsdrucks vom IT-Management:

Viele Organisationen haben ihre Leistungen nach dem ITIL-Service-Lebenszyklus strukturiert und den Ablauf an den ITIL-Prozessen ausgerichtet. ITIL bedingt die Kombination von Zweckmäßigkeit \enquote{Utility} und Einsatzfähigkeit \enquote{Warranty} für einen Wertbeitrag eines Services auf das Geschäft. Zweckmäßigkeit wird durch die Realisierung einer geforderten Funktion oder durch die Beseitigung einer Einschränkung erreicht. Als Qualitätsmerkmal gilt die Einsatzfähigkeit. Sie bezieht sich auf eine ausreichende Verfügbarkeit, Kapazität, Kontinuität und Sicherheit. Es ist Notwendig das bisherige Verständnis von IT-Services um das Kriterium \enquote{Innovationsbeitrag} zu ergänzen. Services müssen sich in einer Innovationskultur bezüglich ihres Beitrags zur Innovationsgenerierung messen lassen.

\cite{Alt2017}


\paragraph{Software as a Service}

\paragraph{Design Thinking}


\paragraph{IT4IT}

\subsection{Methoden und Technologien der IT-Architektur}

\paragraph{Serviceorientierte IT-Architektur}


\paragraph{Infrastructure as a Service}

\paragraph{Cloud-Computing}

\paragraph{Public Cloud}

\paragraph{On-Premise Cloud}

\paragraph{Container-Virtualisierung}

\paragraph{Container-Orchestrierung}

\paragraph{Microservices}