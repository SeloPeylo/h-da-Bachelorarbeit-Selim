%*******************************************************
% Abstract in German
%*******************************************************
\begin{otherlanguage}{ngerman}
	\pdfbookmark[1]{Zusammenfassung}{Zusammenfassung}
	\chapter*{Zusammenfassung}
	Als frühe Anwender von IT-Systemen etablierte sich in Banken eine komplexe IT-Architektur, die regulatorischen Vorgaben unterliegt und regelmäßig geprüft wird. Ansätze und Technologien im Softwareentwicklungsprozess entwickeln sich jedoch ständig weiter. Continuous Integration Plattformen werden mit einer steigenden Frequenz von Build-Aufträgen konfrontiert. Die Plattform delegiert Build-Aufträge an Agenten, die bisher als persistente Virtuelle Maschinen bereitgestellt werden. Ihre Skalierbarkeit ist durch eine unflexible Umgebung und lang anhaltender Prozesse eingeschränkt. Für die Continuous Integration bedingt es eine skalierbare und flexible Umgebung mit neuen Ansätzen, wie Cloud-Computing und Container-Technologie. Mangelnde Skalierbarkeit kann im schlimmsten Fall zur Nichteinhaltung von Schutzzielen wie die Verfügbarkeit der Entwicklungssysteme führen und ist gerade für Betriebe mit hohem Schutzbedarf nicht tragbar.
	\end{otherlanguage}
