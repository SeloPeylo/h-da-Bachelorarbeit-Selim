\chapter{Einleitung}
\label{ch:intro}

Als frühe Anwender von großen IT-Systemen etablierte sich in Banken eine komplexe IT-Architektur \cite{Brockhoff2006}. Daraus formuliert Brockhoff \footnote{stellvertretender Vorsitzender der Finanz Informatik GmbH \& Co KG (Stand 2006)} das Ergebnis, dass in vielen Instituten heterogene, unflexible und eher zufällig gewachsene Architekturen dominieren \cite{Brockhoff2006}.
\medskip
\\
Alle Kreditinstitute und Finanzdienstleistungsinstitute in der Bundesrepublik Deutschland müssen Vorgaben, die aus dem \ac{KWG} hervorgehen auch in ihrer IT umsetzen. Die \ac{BaFin} beaufsichtigt die Einhaltung dieser Vorgaben und stellt eine einheitliche Verwaltungspraxis \cite{BaFin:Verwaltungspraxis} durch ergänzende Anforderungen \cite{MaRisk:2017, BAIT:2018}. 

In der IT-Architektur der Institute entstehen Probleme\footnote{vgl. \cite{mci/Disterer2011}} und einschränkende Faktoren, die nicht nur Auswirkungen auf den Betrieb, sondern auch auf die Entwicklung der IT-Systeme\footnote{IT-Systeme beinhalten Anwendungen, Software, Hardware} haben. 

Für die Entwicklung und Betrieb von IT-Systemen sind in jedem Institut Regelprozesse\footnote{beinhaltet Entwicklung, Testen, Freigabe und Integration \cite{MaRisk:2017}} zu definieren \cite{MaRisk:2017}. 

Zum Beispiel müssen Standardanwendungen von Entwicklern vor ihrem Einkauf nicht nur ausreichend getestet, sondern auch auf die jeweiligen internen Prozesse angepasst werden. Im Ansatz \ac{SaaS} ist die Auslagerung von IT-Ressourcen eine gängige Praxis zum Betreiben von Anwendungen. Ein beliebtes Kollaborationstool wie Slack, dass nach eigenen Angaben über die Cloud von \ac{AWS} zur Verfügung gestellt wird, könnte bereits eine Auslagerung nach §25b des \ac{KWG} darstellen. Dadurch werden zusätzliche Schritte im Risikomanagement der Institute angefordert.
\medskip
\\
Neben den BaFin Anforderungen für ein IT-Risikomanagement und der IT-Architektur selbst ist auch das IT-Management\footnote{vgl. \cite{Alt2017}} eine mögliche Quelle von einschränkenden Faktoren für die Softwareentwicklung im Finanzwesen. Die IT-Architektur ist, bedingt durch Impulse aus neuen Technologien \cite{Bussmann2006}, ständigem Innovationsdruck ausgesetzt und muss sich \enquote{kontinuierlich justieren} \cite{Bussmann2006} können. 
\medskip
\\
Gängige Standards der Softwareentwicklung befinden sich mit neuen Technologien, Ansätzen und Anwendungen im ständigen Wandel. Vor ihrem geplanten Einsatz müssen mögliche Auswirkungen auf die Kontrollverfahren der Institute analysiert werden \cite{MaRisk:2017}.

Die Kredit- und Finanzdienstleistungsinstitute in Deutschland können sich Veränderungen nicht mehr entziehen. Insbesondere ist das aufgrund der Disruptivität von einigen Technologien und Ansätzen bedingt. Eine bevorstehende digitale Transformation von Banken und Finanzdienstleistern bietet für anpassungsfähige Institute einzigartige Möglichkeiten sich neu zu etablieren. Gleichzeitig besteht auch bei Nichtbeachtung von Technologietrends die Gefahr einer Verdrängung von kleineren und effizienten Betrieben, die schnell auf die Impulse reagieren.

Lloyd Blankfein verkündete, als Vorsitzender und \ac{CEO} von Goldman Sachs \emph{\enquote{Wir sind ein Technologieunternehmen. Wir sind eine Plattform.}} \cite{Gupta:2017}. Tatsächlich könnte aus Sicht der Digitalen Transformation eine Bank als Plattform für Finanzdienstleistungen definiert werden und neue Wertschöpfungsketten erzeugen. 

Im Finanzwesen befindet sich möglicherweise auch die deutsche Verwaltungspraxis im Wandel. Insbesondere hat die Cloud-Computing Technologie neue Impulse für das Thema \emph{Auslagerung von IT-Ressourcen\cite{MaRisk:2017, BAIT:2018}} erzeugt. Die MaRisk wird voraussichtlich 2021 erneuert und darüber hinaus sind Erleichterungen aufgrund der Umsetzung der EBA-Richtlinien zu erwarten \cite{BaFin:marisk-novelle}.

Die Anpassung der Verwaltungspraxis kommt jedoch gezwungenermaßen, sodass von einer Disruption gesprochen werden kann.
%
% Section: Motivation
%
\section{Motivation}
\label{sec:intro:motivation}

Die nachfolgende Arbeit ist aus der Frage entstanden, wie man die Softwareentwicklungsumgebung\footnote{Insbesondere die CI Plattform Jenkins} einer Bank skalierbarer und flexibler gestalten kann. Der Autor hat im Rahmen des Praxismoduls der Hochschule Darmstadt ein Praktikum in der DZ BANK AG abgeschlossen.
\medskip
\\
Anfangs wurde der Fokus auf die \ac{CI} Plattform gerichtet und anschließend sich mit Themen wie Cloud-Computing, Container-Orchestrierung und Microservices beschäftigt. Im Praktikum wurde sich mit der Entwicklung von Docker-Images für Jenkins Agenten\footnote{Slave innerhalb einer Master-Slave Konfiguration in Jenkins, vgl. \cite{Pathania2017}} der \ac{CI} Plattform Jenkins beschäftigt. Ziel war hierbei eine flexibel einsetzbare Umgebung zum Ausführen von verteilten Builds\footnote{In diesem Zusammenhang die Erstellung von Software mit dem Build-Tool Maven}.

Jenkins Master verteilt die Build-Jobs aus der Pipeline an Agenten. Die Plattform und ihre Agenten werden in der eigenen Infrastruktur mit persistenten \ac{VMs} betrieben. Die Bestellung, Freigabe und Bereitstellung von konventionellen \ac{VMs} erfolgt hierbei nach einem Regelprozess\cite{MaRisk:2017} mit geringer Geschwindigkeit und Flexibilität. Daraus resultieren Probleme hinsichtlich der Verfügbarkeit der Umgebung, wenn die Ressourcen nicht ausreichen und mehr angefordert werden muss.
\medskip
\\
Ansätze wie Cloud-Computing, \ac{IaaS}, \ac{SaaS}, DevOps und Agilität werden zum Standard für die Softwareentwicklung. Banken benötigen für die Umsetzung von neuen Technologien und Methoden umfangreiche Anpassungen ihrer Prozesse. Selbst eingekaufte Standardanwendungen unterliegen nicht selten umfangreichen Analysen und Anpassungen. 

Es sollte sich bezüglich der internen Kontrollverfahren die Frage gestellt werden, wie mit ständigen Veränderungen in der IT umzugehen ist. Ein angemessenes Risikomanagement sollte Veränderungen fördern. Eine mangelnde Anpassungsfähigkeit und Geschwindigkeit stellt möglicherweise ein höheres Risiko dar.
%
% Section: Ziele
%
\section{Ziel der Arbeit}
\label{sec:intro:goal}
Die Arbeit soll sich mit den Rahmenbedingungen für die Softwareentwicklungsumgebung von Kredit- und Finanzinstituten auseinandersetzen. Darunter zählen unter anderem regulatorische Anforderungen und Vorgaben der \ac{BaFin}\footnote{vgl. \cite{MaRisk:2017, BAIT:2018}}. Zudem werden aus der Literatur bekannte Probleme\footnote{vgl. \cite{Brockhoff2006, Bussmann2006}} ihrer IT-Architektur untersucht. Ziel ist dabei einschränkende Faktoren für Veränderungen zu ermitteln.
\medskip
\\
Es soll untersucht werden, wie die Verwaltungspraxis\footnote{vgl. \cite{BaFin:Verwaltungspraxis}} mit neuen Technologien und Methoden, aus der Impulse für Innovation entstehen \cite{Bussmann2006}, zusammenpassen.
\medskip
\\
Anschließend ist ein weiteres Ziel herauszuarbeiten, was für einen angemessenen Umgang \emph{mit kontinuierlichen Veränderungen der IT-Architektur \cite{Bussmann2006}} und der Softwareentwicklung hinsichtlich neuer Technologien und Methoden bedingt wird.
\medskip
\\
Als nächstes Ziel werden Cloud-Computing und Containervirtualisierung für eine bessere Skalierbarkeit und Flexibilität der Softwareentwicklungsumgebung in Banken untersucht. Daraus soll ein praktischer Ansatz resultieren, dass zeigt wie eine skalierbare und flexible Umsetzung von Jenkins in einer Bank erfolgen könnte. Für dieses Beispiel werden Images für die Jenkins Agenten entwickelt. Diese sollen einen kleinsten gemeinsamen Nenner für die Skalierung der ganzen Plattform darstellen. Das Beispiel soll hinsichtlich der Architektur offen sein, da die Notwendigkeit der Cloud mitdiskutiert werden soll. 
%
% Section: Struktur der Arbeit
%
\section{Aufbau der Arbeit}
\label{sec:intro:structure}
Die Arbeit ist unter drei wesentlichen Bereichen gegliedert. 

Im ersten Bereich ist ein umfangreicher Grundlagenkapitel. Diese fasst bekannte Probleme und Ansätze aus verwandten Werken zusammen und stellt den aktuellen Stand der Technik dar. 

Im zweiten Bereich ist ein umfangreicher Theoriekapitel über die Rahmenbedingungen der Softwareentwicklungsumgebung in Banken. Hierin werden die Einschränkenden Faktoren ermittelt. Zusätzlich werden ihre negativen Auswirkungen und Risiken für das Geschäftsmodell untersucht. Aus der Literaturrecherche werden mögliche Ansätze für eine angemessene Umgebung für die Softwareentwicklung mit aktuellen Technologien und Methoden abgeleitet und erste Lösungsansätze. 

Darüber hinaus wird auf das Innovationspotenzial der IT auf das Geschäftsmodell eingegangen. Die positive Auswirkung von einem Paradigmenwechsel der IT von Banken und Finanzdienstleistern wird anhand von Fallstudien und Beispielen unterstützt.
\medskip
\\
Im dritten Kapitel wird die für die IT relevante Verwaltungspraxis der Bankaufsicht untersucht. Daraus werden wesentlichen regulatorischen Anforderungen für die Entwicklung von nachvollziehbaren Systemen für die \ac{SEU} abgeleitet. Anschließend wird überprüft inwiefern die Verwaltungspraxis die in Kapitel 2 genannten Methoden zulässt und worauf es dabei ankommt. Insbesondere wird der Umgang mit Ständigen Veränderungen in der IT hinterfragt. Zudem wird in diesem Kapitel eine innovationsfördernde Rolle der internen Kontrollverfahren in der IT von Banken gefordert.
\medskip
\\
In vierten Kapitel wird mithilfe der Ergebnisse aus den vorigen Kapiteln ein Ansatz entwickelt, wie eine Continuous Integration Plattform in einer Bank skalierbarer und flexibler betrieben werden kann. Es wird zumindest ein kleinster gemeinsamer Nenner definiert, der die größtmögliche Flexibilität bezüglich der IT-Architektur beibehalten soll. Dazu werden Docker-Images für die Ausführung von Builds entwickelt und mit den nötigen Build-Tools ausgestattet. Die Docker-Images sollen eine sichere und nachvollziehbare Umgebung für einen flexiblen Einsatz in einer Bank bereitstellen. Insbesondere müssen die Softwareartefakte aus dem Erstellungsprozess über ein internes Repository gesichert werden. Zudem sollen die Artefakte, die für die Erstellung des Images verwendet werden aus diesem Repository bezogen werden. 