\chapter{Bewertung und Zusammenfassung}
\label{chapter:evaluation}

\section{Methodik und Ergebnisse}
\emph{Design-Thinking} (Kap. \ref{def-design-thinking}) hat als Methode die vorliegende Arbeit erheblich geprägt. Anfänglich wurde sie beiläufig berücksichtigt und gewann für die Erforschung der Ursachen erheblich an Bedeutung. 
\medskip
\\
Dieses Modell ist auch von Agilität und Kontinuität geprägt, worin sich das Ergebnis ständig verbessert.
Statt einer Iteration kann wieder ganz von vorne begonnen werden. Besonders effektiv ist es, wenn dabei aus dem aktuellen Zustand heraus ein Perspektivwechsel gezielt hin zur Problemquelle stattfindet. Am Ende kann mit einer erheblich größeren Datenmenge eine Synthese aus allen Perspektiven generiert werden als Deduktion. Dabei war die Ausgangssituation von einem sehr eingeschränkten Strategiebild geprägt. Eine problemorientierte Immersion (Kap. \ref{def-design-thinking}) hat das Horizont zielgerichtet und effektiv erweitert. Problemorientierung impliziert schon Kontinuität durch ständigem Verbesserungsbedarf aufgrund neuer Befunde.
Der Design-Thinking Prozess könnte als ein \enquote{Quick-Search} für Innovation gesehen werden. Daneben könnten andere Gestaltungsmethoden ebenfalls zu betrachtet werden.
\medskip
\\
Sie hinterlässt jedoch Spuren\footnote{Anhang: \label{appendix:docker}} und es Besteht die Gefahr zu divergieren, da sie mit einer erheblichen Geschwindigkeit induktiv vorgeht und auf kontinuierliche Verbesserung der Prototypen beruht. Für die Qualitätssicherung der Lösung verlangt es einen effizienten Vorgang der Integration der Komponenten und Implementierung einer Lösung mit entsprechenden Methoden.

\subsection{Von CI/CD zu Change-Management}
Die Anfangssituation aus der \ac{SEU} in Kap. \ref{grundlagen:ci-in-banken} wurde bis zu einer unzumutbaren Hindernissschwelle bearbeitet und erste Lösungsansätze mit gängigen Standards beschrieben. Die vorgeschlagenen Ansätze lösen jedoch nicht den Ursprung des Problems, da sie umfassende Auswirkungen und Anpassungen nach sich ziehen, die eine Transformation der ganzen Organisation fordern. 

Für die Ursache des Problems fand ein Wechsel der Perspektive in Kap. \ref{section:Goldman} zu einem Wettbewerber statt, der eine Transformation bereits durchgemacht hat. Diese Transformation ist als Fallbeispiel von \citet{Gupta:2017} empirisch beschrieben und enthält wiederum verschiedene Stellungnahmen aus der Managementebene.

In den Schritten \emph{empathize}, \emph{define} und \emph{ideate} wird die Ursache des Problems analysiert möglichst ohne vorherige Annahme zur Quelle des Problems. Wichtig ist, dass hierbei wirklich die Perspektive gewechselt wird und eine Immersion in die Umgebung des Kunden stattfindet. Daher kann diese Gruppe statt Analyse \cite{yüksel:digit} auch Immersion genannt werden. 
\medskip
\\
Die Schritte zu \emph{Synthese} sind eine Datenerhebung \cite[S. 61]{yüksel:digit} durch Gestaltung von Zukunftszuständen, die nach \citet[S. 14]{Alt2017} für eine \emph{Vorwegnahme künftiger Innovationen} von besonderer Bedeutung ist. 

Die Opportunitäten durch Goldman Sachs Plattform \cite{Gupta:2017} und der Hintergrund von disruptiven Technologien \cite{Fernandez:2020} wurde zum Anlass genommen Ideen zu formulieren. In Kap. \ref{section:Goldman} wurden weitere Zukunftszustände gestaltet.
Wichtig ist hierbei, dass ungezwungen so viele Daten wie möglich generiert und erhoben werden. Mit dem richtigen Verständnis aus den vorigen immersiven Schritten kann eine Deduktion nach den Anforderungen des Kunden stattfinden. 
\medskip
\\
Die erhobenen Daten waren anfänglich neue Technologien zur Lösung des Problems und Einschränkungen im Rahmen der IT-Architektur. Das Problem hierbei ist, dass ab einer gewissen Größe des Unternehmens sie sich gegenseitig voraussetzen. 

Daher wurde früh erkannt, dass die Ursache in der Transformationsfähigkeit der Organisation liegen könnte. Aus einer gemeinsamen Betrachtung der Sicht der \ac{SEU} \ref{grundlagen:ci-in-banken} und Goldman Sachs \cite{Gupta:2017} spielten hierfür die regulatorischen Anforderungen als Einschränkung und Kultur als Antrieb eine wichtige Rolle. 

Der Perspektivwechsel mit der \emph{Design-Thinking} Methode \ref{def-design-thinking} durch eine Iterationen in den anfänglichen Empathize Zustand ermöglichte die Sicht der Regulatorik auf Technologien und Veränderungen zu betrachten und ist daher für die Generierung von weiteren Ergebnissen wichtig, bis die Lösung der Problemursache gefunden wurde.
\section{Einschränkende und Antreibende Faktoren}
\label{section:faktoren-change}

\paragraph{Einschränkende Faktoren der IT für Veränderungen}
\begin{enumerate}
    \item 
\end{enumerate}

\paragraph{antreibende Faktoren der IT für Veränderungsprozesse}
Die nachfolgenden antreibenden Faktoren betreffen die Fähigkeit der IT zur Umsetzung ihrer Veränderungs- oder \enquote{Change-} Prozesse. Diese sind nachfolgend mit einem allgemeinen Begriff aufgezählt und enthalten eine Analye und Bewertung. Die Auflistung ist anschließend sortiert nach der gegenseitigen Voraussetzung der Punkte. Ein genannter Punkt wird unter anderem für den Folgepunkt vorausgesetzt.

\begin{enumerate}
    \item \textbf{Offenheit \cite{Brockhoff2006}}
    
    \item \textbf{Zusammenarbeit}

    \item \textbf{Standardisierung} 
    
    Dieser Punkt umfasst Anwendungen, Systeme und Prozesse. Die Verwendung von gängigen sowie standardisierten Lösungen bringt eine Reihe von Vorteilen. Sie erhöht allen voran die Nachvollziehbarkeit der Systeme und sorgt für eine Effizienzsteigerung von Prozessen \cite{Strietzel2018, Bussmann2006, Alt2017}. 
    
    Beteiligte, insbesondere die Prüfer der Bankaufsicht und die interne Revision, können so mit einem Grundwissen über die IT die meisten Systeme auf Anhieb verstehen. Die Implementierung von \emph{Sicherheitstandards \cite{IT-Grundschutz:2020, Disterer2013}} ist hierbei besonders wichtig und bietet eine Grundlage für die Einhaltung der regulatorischen Vorgaben \cite{MaRisk:2017, BAIT:2018}.
    
    Der Einsatz von Standardsoftware führt zu einem effizienteren Ablauf in ihrer Wartung und Weiterentwicklung \cite{Bussmann2006}. Die Komplexität von operativen Systemen schränkt die Einführung von Standardsoftware jedoch ein \cite[S.27]{Bussmann2006}. Das Prinzip der \emph{Standardisierung} in Verbindung mit \emph{Offenheit} ergibt die Verwendung von \enquote{Open-Source} Software, die eine höhere Effizienzsteigerung aufgrund der Wiederverwendbarkeit der Komponenten bezweckt \cite{Brockhoff2006, Gupta:2017}. 
    
    Die Einführung von Standardsoftware hat ebenfalls Auswirkungen auf eingespielte Prozesse und die daraus folgenden Einzelentscheidungen beeinflussen in ihrer Summe die Entscheidung für oder gegen den Standard \cite[Tab.1]{Manz2018}.
    
    \item \textbf{Modularisierung}
    
    \item \textbf{Nachvollziehbarkeit}
    
    \item \textbf{Automatisierung}
    
    \item \textbf{Effizienzsteigerung}
    
    Die Effizienzsteigerung wirkt sich auf die verwendeten Ressourcen und Zeitaufwand der Prozesse, Systeme oder Anwendungen aus. Dazu zählen auch Kosten- und Personalaufwand. Es stellt sich die Frage in welchen Bereichen eine Effizienzsteigerung am sinnvollsten ist.
    
    Die Effizienzsteigerung wird vor allem im Bereich der Veränderungsprozesse \enquote{Change} bezüglich ihres Zeitaufwands erforderlich. Im Bereich der betrieblichen Prozesse \enquote{Run} ist sie primär für den Aufwand der Ressourcen erforderlich.
    
    Eine Effizienzsteigerung im Betrieb ermöglicht mehr Mittel für die \enquote{Change} Prozesse \cite{Rausch2006}. Dagegen steht das Prinzip, dass Ressourcen im Gegensatz zu Zeit finanziert und eingekauft werden können. Bevor an Opportunitäten in Form einer strategischen Weiterentwicklung der Produkte \cite{Rausch2006} gedacht wird, sollte die Weiterentwicklung der hierfür vorausgesetzten Veränderungsprozesse berücksichtigt werden. Die Fähigkeit für eine kontinuierliche Anpassung \cite{Bussmann2006, Ganswindt2006} der Produkte wird mit aktuellen Methoden, wie in Abb. \ref{fig:devops} dargestellt, bereits vorausgesetzt \cite{Alt2017}.

\end{enumerate}

\paragraph{Innovationsverlauf}

\paragraph{Prototyp: Regelung der Innovation}
Im Rahmen des Innovationsdrucks vom IT-Management gibt es Rahmenwerke für Prozesse, wie zum Beispiel ITIL \ref{def:itil}, worin Services sich bezüglich ihres Innovationsbeitrags messen lassen sollen \cite{Alt2017}. Im Rahmen dieser Arbeit wurden die Frameworks wie ITIL oder auch das offene Framework IT4IT wahrgenommen, jedoch nicht weiter untersucht. Im Rpckblick kann festgestellt werden, dass viel mehr ein klares Bild für das was ITIL bezwecken soll vonnöten ist, vor allem für das \ac{ITM}.
\medskip
\\
Es kann festgestellt werden, dass die Darstellung \ref{fig:digit-trans} einen Beitrag zur Messbarkeit für Innovationskraft eines IT-Produktes leistet, um wiederum für eine IT-Strategie zu priorisieren. Daraus wird klar, dass antreibende Technologien eine Anlage für die Generierung weiterer Innovationen sind. Die antreibenden Faktoren, die aus dem Ziel der Maximierung der Transformationsbewegung generiert wurden stellen möglicherweise eine Anforderung für Antriebe oder antreibenden Anlagen dar. Hieraus könnten weitere Ansätze generiert werden, um gezielt die Effektivität einer Innovation zu messen. Diese ist wichtig für eine Priorisierung seitens \ac{ITM}.

\paragraph{Prototyp: Steuerung der Innovation}


\section{Zusammenfassung und Ausblick}
Es ergibt sich ein Gesamtbild, dass die aufgeführten Faktoren für Veränderungsprozesse keinen Ablauf oder direkte Handlungsanweisungen darstellen. Diese leiten sich aus dem gegenwärtigen Zustand ab und beschreiben \emph{Einflüsse} für Veränderungsprozesse und haben wesentliche Auswirkungen auf ihren Ablauf. Insbesondere die antreibenden Faktoren für Veränderungsprozesse können als ihre fachlichen Anforderungen gesehen werden und die Einschränkungen als ihre Risiken. Kontinuierliche Ansätze wie \emph{DevOps} und \emph{Design-Thinking} helfen dabei die Veränderungsprozesse selbst agil und kontinuierlich anzupassen. 
\medskip
\\
Darüber hinaus wird deutlich, dass die Informatik weitreichende Kompetenzen verfügt, aus dem Zusammenhang heraus auch Probleme aus anderen Bereichen zu lösen. Beginnend von der Ausgangssituation eines vorwiegend technischen Problems haben innovationsorientiere Ansätze weitreichende Implikationen auf ihre nicht-technische Umgebung. Die Abkürzung IT sollte im Sinne einer Organisation künftig nicht für Informationstechnologie stehen sondern für \enquote{Innovation und Transformation}, wodurch es Bezug auf die Eingangs- und Ausgangsströme des Paradigmas \enquote{Innovate-Design-Transform} nimmt. Dazwischen liegt unser Ergebnis, das Akkumulieren als Wertschöpfung.
\medskip
\\
Kreative Methoden sind durch die ad-hoc Generierung von Prototypen oder Beschreibungen wichtige Mittel für die Gestaltung von immateriellen Zuständen und daher für Innovationsprozesse geeignet. Die Synthese ihrer Ergebnisse ergibt sich aus der Kohärenz zwischen gegenwärtigem Zustand, zukünftigen Risiken sowie Opportunitäten der verschiedenen Perspektiven. 
\medskip
\\
\citet{Ganswindt2006} beschreibt:
\begin{quote}
    \enquote{
    Ein Unternehmen der Wissensgesellschaft muss sich deshalb so organisieren, dass die Ressource Wissen zu einer steten Quelle eines Ideenstroms wird – egal, von welchem Ort der Welt aus der Mitarbeiter tätig ist. Es ist ein Muss, bei den Mitarbeitern Begeisterung für neue Ideen zu wecken, das Vertrauen in die eigene Kreativität zu fördern und ihnen dann Raum zu lassen, das Neue auch umzusetzen und ihr Wissen zu teilen
    }
\end{quote}
\medskip
Sie bieten keine direkten Anweisungen zur Implementierung einer Lösung, sind aber für ihre Strategie besonders wertvoll. Die steigende Kontinuität, Geschwindigkeit und Auswirkung von Veränderungen verlangen das Treffen von schnellen Entscheidungen, wofür Agilität und Kreativität in der Strategie bedingt wird. Geschieht dies nicht kann auch kein effektiver Veränderungsprozess erzeugt werden, woraus Ziele der Strategie erfolgreich umgesetzt werden. Mit ineffizienten Innovationsprozessen entsteht für die Transformation eine Paralyse durch Analyse, mit darauf folgender Handlungsunfähigkeit. 

Für eine effektive Transformation bedingt es reibungslose Veränderungsprozesse, welche die Transformationsfähigkeit der Umgebung messen können. Danach sollte die Entwicklung von Gegenmaßnahmen als Antreibende Anlagen für die Transformationsfähigkeit priorisiert werden. Als einer der einflussreichsten Maßnahmen fallen hierunter Plattformen für die IT und Anwendungen für die Unterstützung von Steuerungs- und Überwachungsprozessen.

Das Kontinuitätsprinzip hat bereits in vielen Bereichen der Softwareentwicklung eine Grundlage für effiziente und effektive Vorgehen geschaffen. Beispiele wären \ac{CI} für die Integration von Komponenten und \ac{CD} für ein effektives und kontrolliertes Ausrollen in die Umgebung. Zusammengefasst in DevOps ergeben sie einen kontinuierlichen Lebenszyklus der Verbesserung von IT-Produkten. Das gleiche Prinzip kann auch auf das Prozessmanagement angewandt werden.

Die Digitale Transformation erzeugt viel Ungewissenheit mit darauf folgender Handlungsunfähigkeit. Das Problem an der Partizipation sind die Eintrittshürden für Organisationen mit Legacy-IT. Jedem Versuch einer Transformation widerstehen die gegenseitigen Abhängigkeiten der Systeme und Prozesse. Organisationen beginnen die Transformation an falscher Stelle, wodurch Projekte eingestellt werden. Daher wird auch ein kontinuierlicher Lebenszyklus der Verbesserung von Veränderungsprozessen bedingt. Ein Change-Management, dass mithilfe eines starken Innovationsmanagements Anlagen für den Antrieb der Transformationsfähigkeit identifiziert und fördert und somit kontinuierlich den Innovationsmotor beschleunigt.
Das neue Paradigma \emph{innovate-design-transform} könnte mit dem zusätzlichen Punkt \emph{aggregate} ein Kontinuum formen und Grundlage für kontinuierliche Verbesserungen der Veränderungsprozesse sein. Ein erweitertes Modell mit Metriken für Innovationsfähigkeit, Gestaltungsfähigkeit, Transformationsfähigkeit

\paragraph{Ausblick}


Die Umwandlung von sequenziellen und aufwändigen Rollouts für Changes in eine vollautomatisierte Orchestrierung mit kontinuierlicher Transformationsfähigkeit der Organisationen könnte womöglich neben der für die IT revolutionären Cloud-Plattformen eine viel größere Auswirkungen nach sich ziehen. 

Zukünftig könnten Organisationen sich mit einem kontinuierlichen \enquote{Innovate-Transform} Lebenszyklus eine unvorstellbare Schnelligkeit aufbauen.
Die Weiterentwicklung des \emph{innovate-design-transform} Paradigmas würde die Effizienz von Transformationsprozessen durch Effektivität von Innovationsprozessen mit einem entsprechend inkrementellen Antrieb steigern. Die Transformationsfähigkeit als kulturelles Prinzip könnte eine von Grund auf agile Organisation nach sich ziehen. Entsprechend würden umfassende Tätigkeiten ausgelagert oder konsolidiert werden, sodass die Kernkompetenz eines Unternehmens zukünftig aus Innovationskraft und Transformationsfähigkeit besteht. \enquote{Innovation und Transformation} verschafft der IT somit eine neue Bedeutung.