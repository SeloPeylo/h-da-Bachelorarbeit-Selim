\chapter{Analyse und Bewertung}


\section{Einschränkende und Antreibende Faktoren}
\label{section:regulatorik-umgang}

\paragraph{antreibende Faktoren der IT für Veränderungen}
Die nachfolgenden antreibenden Faktoren betreffen die Fähigkeit der IT zur Umsetzung ihrer Veränderungs- oder \enquote{Change-} Prozesse. Diese sind nachfolgend mit einem allgemeinen Begriff aufgezählt und enthalten eine detaillierten Erklärung und Argumentation. Die Auflistung ist in ihrer Reihenfolge sortiert. Einige Punkte werden für die Folgepunkte vorausgesetzt.

\begin{enumerate}
    \item \textbf{Offenheit}
    
    \item \textbf{Zusammenarbeit}

    \item \textbf{Standardisierung} 
    
    Dieser Punkt umfasst viele Aspekte von Systemen bis hin zu Prozessen. Die Verwendung von gängigen sowie standardisierten Lösungen bringt eine Reihe von Vorteilen. Sie erhöht allen voran die Nachvollziehbarkeit der Systeme und sorgt für eine Effizienzsteigerung von Prozessen \cite{Strietzel2018, Bussmann2006, Alt2017}. 
    
    Beteiligte, insbesondere die Prüfer der Bankaufsicht und die interne Revision, können so mit einem Grundwissen über die IT die meisten Systeme verstehen. Die Implementierung von \emph{Sicherheitstandards \cite{IT-Grundschutz:2020, Disterer2013}} ist hierbei besonders wichtig und bietet eine Grundlage für die Einhaltung der regulatorischen Vorgaben \cite{MaRisk:2017, BAIT:2018}.
    
    Der Einsatz von Standardsoftware führt zu einem effizienteren Ablauf in ihrer Wartung und Weiterentwicklung \cite{Bussmann2006}. Die Komplexität von operativen Systemen schränkt die Einführung von Standardsoftware jedoch ein \cite{Bussmann2006, S. 27}. 
    
    Die Einführung von Standardsoftware hat ebenfalls Auswirkungen auf eingespielte Prozesse und die daraus folgenden Einzelentscheidungen beeinflussen in ihrer Summe die Entscheidung für oder gegen den Standard \cite{Manz2018}.
    
    \item \textbf{Modularisierung}
    
    \item \textbf{Nachvollziehbarkeit}
    
    \item \textbf{Automatisierung}
    
    \item \textbf{Effizienzsteigerung}
\end{enumerate}


\paragraph{Einschränkende Faktoren der IT für Veränderungen}
\begin{enumerate}
    \item 
\end{enumerate}


\section{Zusammenfassung und Ausblick}