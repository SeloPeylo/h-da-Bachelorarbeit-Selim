Die Bereitstellung von neuen persistenten virtuellen Maschinen auf der hauseigenen Infrastruktur durchläuft einen langen Prozess, der vorwiegend manuell abgearbeitet wird. Im schlimmsten Fall kann die mangelnde Fähigkeit zur Skalierung von Anwendungen, in ressourcenintensiven Fällen, wie der Softwareerstellungsprozess, die Verfügbarkeit von Systemen beeinträchtigen. Daher wird mit der Google Cloud Plattform und Kubernetes nach einer einfachen Lösung gesucht die CI Plattform skalierbar und flexibel zu betreiben. Die Google Kubernetes Engine bietet für Container automatische Skalierung und Load-Balancing, weshalb der Betrieb von Jenkins in einem Kubernetes Cluster in Frage kommt. Holistisch gesehen muss für die Nutzung der Google Cloud Plattform mit der hauseigenen Architektur ein hoher Maß an Anpassungen stattfinden, die aufgrund von internen und regulatorischen Anforderungen erfordert wird. Alternativ kann mit Minikube ein Kubernetes Cluster auch selbst betrieben werden.
Ein zu hoher Maß an Anpassungen, der einer Kompensation für eine unflexible IT-Architektur oder Ablauforganisation gleicht, könnte die Vorteile von neuen Technologien jedoch negieren. Hierbei ist oft unklar, ob Anpassungen wirklich auf regulatorische Anforderungen zurückzuführen sind oder einfach aufgrund der verwachsenen unflexiblen IT-Architektur in Banken erfordert ist. Speziell die Container-Technologie bietet die Möglichkeit unbefangen an einem Konzept mit den tatsächlichen regulatorischen Anforderungen zu arbeiten, da Container oder ganze Container-Architekturen mit Kubernetes plattformunabhängig sind.


Die Maßnahmen sollten Anwendungen für die IT von Anwendungen für andere Fachbereiche unterschieden.

Umgekehrt ist auch ein IT-Verständnis für die Verantwortlichen der Strategie und Kontrollverfahren wichtig, um die Einführung neuer Methoden für die IT zu begünstigen. 
Daher sollten Entwickler in die Strategie und Kontrollverfahren der Institute integriert werden.


Innovative IT für Banken - Entkernung eines Systems mit Containern und Cloud\\
Moderne Methoden und Technologien für die IT von Banken und ein Ansatz für eine skalierbare und flexible CI Plattform \\
Continuous Integration in Banken - Skalierbarkeit und Flexibilität der Softwareentwicklungsumgebung\\
Continuous Integration in Banken - Skalierbarkeit und Flexibilität von automatisierten Builds\\
Die IT der Kreditinstitute - Skalierbarkeit und Flexibilität am Beispiel einer Continuous Integration Plattform


Ein Anwendungsbeispiel der Cloud wäre die automatisierte Skalierung von Anwendungen und Systemen.
Probleme bezüglich der Skalierung von Anwendungen im Betrieb sind kein neues Problem. Zu  wie Jenkins existieren öffentliche Ansätze \ref{fig:gkejenkins}, die auf den jeweiligen Anwendungsfall angepasst werden können.