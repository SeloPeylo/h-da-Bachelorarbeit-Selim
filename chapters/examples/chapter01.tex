\chapter{Einleitung}
\label{ch:intro}

\citet{Brockhoff2006} formuliert das Ergebnis, dass in vielen Banken heterogene, unflexible und eher zufällig gewachsene Architekturen dominieren. Das Ergebnis leitet er aus dem historischen Kontext ab, dass Banken frühe Anwender von IT-Systemen sind, woraus sich eine komplexe IT-Architektur etablierte.
\citet{Strietzel2018} fasst das Ergebnis aus der veralteten IT-Architektur von Banken mit der Bezeichnung \emph{Legacy-IT} zuzsammen.
\medskip
\\
Alle Kreditinstitute und Finanzdienstleistungsinstitute in der Bundesrepublik Deutschland müssen Vorgaben, die aus dem \ac{KWG} hervorgehen auch in ihrer IT umsetzen. Die \ac{BaFin} beaufsichtigt die Einhaltung dieser Vorgaben und stellt eine einheitliche Verwaltungspraxis \cite{BaFin:Verwaltungspraxis} durch ergänzende Anforderungen \cite{MaRisk:2017, BAIT:2018}. 

In der IT-Architektur der Institute entstehen Probleme\footnote{vgl. \cite{mci/Disterer2011}} und einschränkende Faktoren, die nicht nur Auswirkungen auf den Betrieb, sondern auch auf die Entwicklung der IT-Systeme\footnote{IT-Systeme beinhalten Anwendungen, Software, Hardware} haben. 

Für die Entwicklung und Betrieb von IT-Systemen sind in jedem Institut Regelprozesse\footnote{beinhaltet Entwicklung, Testen, Freigabe und Integration \cite{MaRisk:2017}} zu definieren \cite{MaRisk:2017}. 

Zum Beispiel müssen Standardanwendungen von Entwicklern vor ihrem Einkauf nicht nur ausreichend getestet, sondern auch auf die jeweiligen internen Prozesse angepasst werden. Im Ansatz \ac{SaaS} ist die Auslagerung von IT-Ressourcen eine gängige Praxis zum Betreiben von Anwendungen. Ein beliebtes Kollaborationstool wie Slack, dass nach eigenen Angaben über die Cloud von \ac{AWS} zur Verfügung gestellt wird, könnte bereits eine Auslagerung nach §25b des \ac{KWG} darstellen. Dadurch werden zusätzliche Schritte im Risikomanagement der Institute angefordert.
\medskip
\\
Gängige Standards der Softwareentwicklung befinden sich mit neuen Technologien, Ansätzen und Anwendungen im ständigen Wandel. Vor ihrem geplanten Einsatz müssen mögliche Auswirkungen auf die Kontrollverfahren der Institute analysiert werden \cite{MaRisk:2017}.
Die IT-Architektur ist, bedingt durch Impulse aus neuen Technologien \cite{Bussmann2006}, ständigem Innovationsdruck ausgesetzt und muss sich \enquote{kontinuierlich justieren} \cite{Bussmann2006} können. 
\medskip
\\
Die Kredit- und Finanzdienstleistungsinstitute in Deutschland können sich Veränderungen nicht mehr entziehen. Insbesondere ist das aufgrund der Disruptivität von einigen Technologien und Ansätzen bedingt. Eine bevorstehende digitale Transformation von Banken und Finanzdienstleistern bietet für anpassungsfähige Institute einzigartige Möglichkeiten sich neu zu etablieren. Gleichzeitig besteht auch bei Nichtbeachtung von Technologietrends die Gefahr einer Verdrängung von kleineren und effizienten Betrieben, die schnell auf die Impulse reagieren.

Lloyd Blankfein verkündete, als Vorsitzender und \ac{CEO} von Goldman Sachs \emph{\enquote{Wir sind ein Technologieunternehmen. Wir sind eine Plattform.}} \cite{Gupta:2017}. Tatsächlich könnte aus Sicht der Digitalen Transformation eine Bank als Plattform für Finanzdienstleistungen definiert werden und neue Wertschöpfungsketten erzeugen. 
\medskip
\\

Die Anpassung von bewährten Standardlösungen auf eine problematische Umgebung \cite{Bussmann2006, Brockhoff2006, Strietzel2018} lindert oftmals nur die Symptome und ist, gerade in den Systemen der Finanzinstitute mit hohem Aufwand verbunden. Ein grundlegendes Re-Design \cite{Bussmann2006} ist durch Cloud-Technologie realistischer geworden. Hierfür entstehen weitere Probleme die aus den externen Rahmenbedingungen rühren. Als eine Einschränkung können die regulatorischen Anforderungen \cite{MaRisk:2017} angesehen werden. Nach der Betrachtung von diesen kann festgestellt werden, dass aus Sicht eines Kreditinstiuts eine Wechselwirkung zwischen Veränderungsdruck und Widerstand herrscht. Eine wichtige Frage ist daher, worin die Ursachen für eine mangelnde Durchsetzungsfähigkeit für Veränderungen liegen.
%
% Section: Motivation
%
\section{Motivation}
\label{sec:intro:motivation}

Die nachfolgende Arbeit ist aus der Frage entstanden, wie man die Softwareentwicklungsumgebung\footnote{Insbesondere die CI Plattform Jenkins} einer Bank skalierbarer und flexibler gestalten kann. Der Autor hat im Rahmen des Praxismoduls der Hochschule Darmstadt ein Praktikum in der DZ BANK AG abgeschlossen.
\medskip
\\
Anfangs wurde der Fokus auf die \ac{CI} Plattform gerichtet und anschließend sich mit Themen wie Cloud-Computing, Container-Orchestrierung und Microservices beschäftigt. Im Praktikum wurde sich mit der Entwicklung von Docker-Images für Jenkins Agenten\footnote{Slave innerhalb einer Master-Slave Konfiguration in Jenkins, vgl. \cite{Pathania2017}} der \ac{CI} Plattform Jenkins beschäftigt. Ziel war hierbei eine flexibel einsetzbare Umgebung zum Ausführen von verteilten Builds\footnote{In diesem Zusammenhang die Erstellung von Software mit dem Build-Tool Maven}.

Jenkins Master verteilt die Build-Jobs aus der Pipeline an Agenten. Die Plattform und ihre Agenten werden in der eigenen Infrastruktur mit persistenten \ac{VMs} betrieben. Die Bestellung, Freigabe und Bereitstellung von konventionellen \ac{VMs} erfolgt hierbei nach einem Regelprozess\cite{MaRisk:2017} mit geringer Geschwindigkeit und Flexibilität. Daraus resultieren Probleme hinsichtlich der Verfügbarkeit der Umgebung, wenn die Ressourcen nicht ausreichen und mehr angefordert werden muss.
\medskip
\\
Ansätze wie Cloud-Computing, \ac{IaaS}, \ac{SaaS}, DevOps und Agilität werden zum Standard für die Softwareentwicklung. Banken benötigen für die Umsetzung von neuen Technologien und Methoden umfangreiche Anpassungen ihrer Prozesse. Selbst eingekaufte Standardanwendungen unterliegen nicht selten umfangreichen Analysen und Anpassungen. 

Es sollte sich bezüglich der internen Kontrollverfahren die Frage gestellt werden, wie mit ständigen Veränderungen in der IT umzugehen ist. Ein angemessenes Risikomanagement sollte Veränderungen fördern. Eine mangelnde Anpassungsfähigkeit und Geschwindigkeit stellt möglicherweise ein höheres Risiko dar.
%
% Section: Ziele
%
\section{Ziel der Arbeit}
\label{sec:intro:goal}

Die nachfolgende Arbeit soll sich mit den Hintergründen der bereits länger bekannten Einschränkungen für den \emph{Betrieb} der IT von Banken beschäftigen \cite{Brockhoff2006},  \cite{Bussmann2006}, \cite{Gupta:2017}.
Diese Einschränkungen haben Auswirkungen auf die \ac{SEU} in der IT von Banken und es stellt sich die Frage:
\begin{quote}
    \enquote{Welche Ansätze eignen sich für eine skalierbare und flexible Softwareentwicklungsumgebung in Banken?}
\end{quote}
%
Die Einführung von bekannten Methoden\footnote{Beispiele: Design-Thinking, Scrum, DevOps, \ac{SOA}, Microservices, Cloud} und Rahmen\footnote{Beispiele: ITIL, IT4IT} für die Umsetzung von Veränderungsprozessen erfolgt im Finanzwesen jedoch eingeschränkt und verspätet. Für ihre Einführung unterliegen diese Ansätze umfangreichen Anpassungen oder werden durch neue Technologien und Trends wieder überflüssig.
\medskip
\\
Daher wird als Voraussetzung für die Lösung dieser Forschungsfrage bedingt, einschränkende und antreibende Faktoren für die \emph{Veränderung} innerhalb der IT von Banken zu identifizieren. Eine zweite Forschungsfrage:
\begin{quote}
    \enquote{Welche einschränkenden und antreibenden Faktoren gibt es für die IT von Banken, um Veränderungen durchzuführen?}
\end{quote}

Dazu werden die Rahmenbedingungen für Veränderungen aus verschiedenen Perspektiven durchleuchtet:
\begin{itemize}
    \item \emph{Finanzwesen}, anhand einer Fallstudie \cite{Gupta:2017} über Goldman Sachs' Transformation
    \item \emph{Technologietrends}, die Impulse geben und später Disruptivität oder Diskontinuität erzeugen könnten
    \item \emph{Bankaufsicht}, die mit ihren Vorgaben \cite{MaRisk:2017}, \cite{BAIT:2018} einen Rahmen für das Finanzwesen festlegt, mit Auswirkung auf die IT
\end{itemize}
 
Aus dem genannten Zusammenhang werden die gegenseitigen Einflüsse identifiziert und zwischen einschränkende und antreibende Faktoren für Veränderungsprozesse unterschieden und gegenüber gestellt. 
\medskip
\\
\emph{Veränderungen} sind aufgrund der Auswirkungen von Technologietrends unvermeidbar und müssen mittlerweile kontinuierlich erfolgen \cite{Bussmann2006}, \cite{Alt2017}, \cite{Fernandez:2020}. Daher stellt sich die Forschungsfrage: 
\begin{quote}
    \enquote{Wie sollten Banken bezüglich ihrer IT mit kontinuierlichen Veränderungen umgehen?}
\end{quote}

Die Arbeit soll sich auch mit einem Paradigmawechsel der IT im Finanzwesen beschäftigen. Mithilfe der antreibenden Faktoren werden die Einschränkungen in einem Deduktionsschritt entkräftigt. Die übrig bleibenden Mindesteinschränkungen für Veränderungsprozesse dienen als Grundlage für die Evaluation der Ansätze aus der ursprünglichen Forschungsfrage. Somit finden sich möglicherweise geeignete Ansätze, für eine skalierbare und flexible Softwareentwicklungsumgebung in Banken.


%
% Section: Struktur der Arbeit
%
\section{Aufbau der Arbeit}
\label{sec:intro:structure}
Das Problem wird zuerst aus der Sicht der \ac{SEU} definiert und von unten aus auf die allgemeinen Ursachen verfolgt. Von oben aus werden die Hintergründe aus unterschiedlichen Perspektiven durchleuchtet und wesentliche Punkte deduktiv ausgearbeitet und bewertet.

Die Arbeit verwendet gleichzeitig die meisten Schritte aus dem \emph{Design-Thinking} Prozess. Daher werden schon früh induktiv einfache Ansätze für erste Lösungen ad hoc vorgeschlagen und beschrieben. Dieses frühzeitige Prototyping in Kap. \ref{ch:background} bezweckt, dass Ansätze und Anforderungen von der angenommenen Perspektive heraus vorgeschlagen werden. Diese können sich somit gegenseitig ausschließen, sodass bereits eine Filterung vor der eigentlichen Evaluation in \ref{chapter:evaluation} stattfindet. In diese fließen anschließend die festgehaltenen Ergebnisse.

Die Arbeit ist unter drei wesentlichen Bereichen gegliedert. 
Im ersten Bereich ist ein umfangreicher Grundlagenkapitel. Diese fasst bekannte Probleme und Ansätze aus verwandten Werken zusammen und stellt den aktuellen Stand der Technik dar. 

Im zweiten Bereich ist ein umfangreicher Theoriekapitel über die Rahmenbedingungen der Softwareentwicklungsumgebung in Banken. Hierin werden die Einschränkenden Faktoren ermittelt. Zusätzlich werden ihre negativen Auswirkungen und Risiken für das Geschäftsmodell untersucht. Aus der Literaturrecherche werden mögliche Ansätze für eine angemessene Umgebung für die Softwareentwicklung mit aktuellen Technologien und Methoden abgeleitet und erste Lösungsansätze. 

Darüber hinaus wird auf das Innovationspotenzial der IT auf das Geschäftsmodell eingegangen. Die positive Auswirkung von einem Paradigmenwechsel der IT von Banken und Finanzdienstleistern wird anhand von Fallstudien und Beispielen unterstützt.
\medskip
\\
Im dritten Kapitel wird die für die IT relevante Verwaltungspraxis der Bankaufsicht untersucht. Daraus werden wesentlichen regulatorischen Anforderungen für die Entwicklung von nachvollziehbaren Systemen für die \ac{SEU} abgeleitet. Anschließend wird überprüft inwiefern die Verwaltungspraxis die in Kapitel 2 genannten Methoden zulässt und worauf es dabei ankommt. Insbesondere wird der Umgang mit Ständigen Veränderungen in der IT hinterfragt. Zudem wird in diesem Kapitel eine innovationsfördernde Rolle der internen Kontrollverfahren in der IT von Banken gefordert.
\medskip
\\
In vierten Kapitel wird mithilfe der Ergebnisse aus den vorigen Kapiteln ein Ansatz entwickelt, wie eine Continuous Integration Plattform in einer Bank skalierbarer und flexibler betrieben werden kann. Es wird zumindest ein kleinster gemeinsamer Nenner definiert, der die größtmögliche Flexibilität bezüglich der IT-Architektur beibehalten soll. Dazu werden Docker-Images für die Ausführung von Builds entwickelt und mit den nötigen Build-Tools ausgestattet. Die Docker-Images sollen eine sichere und nachvollziehbare Umgebung für einen flexiblen Einsatz in einer Bank bereitstellen. Insbesondere müssen die Softwareartefakte aus dem Erstellungsprozess über ein internes Repository gesichert werden. Zudem sollen die Artefakte, die für die Erstellung des Images verwendet werden aus diesem Repository bezogen werden. 