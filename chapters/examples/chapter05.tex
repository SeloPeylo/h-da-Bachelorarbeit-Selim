\chapter{Bewertung und Zusammenfassung}
\label{chapter:evaluation}

\section{Einschränkende und Antreibende Faktoren}
\label{section:faktoren-change}

\paragraph{Einschränkende Faktoren der IT für Veränderungen}
\begin{enumerate}
    \item 
\end{enumerate}

\paragraph{antreibende Faktoren der IT für Veränderungsprozesse}
Die nachfolgenden antreibenden Faktoren betreffen die Fähigkeit der IT zur Umsetzung ihrer Veränderungs- oder \enquote{Change-} Prozesse. Diese sind nachfolgend mit einem allgemeinen Begriff aufgezählt und enthalten eine Analye und Bewertung. Die Auflistung ist anschließend sortiert nach der gegenseitigen Voraussetzung der Punkte. Ein genannter Punkt wird unter anderem für den Folgepunkt vorausgesetzt.

\begin{enumerate}
    \item \textbf{Offenheit \cite{Brockhoff2006}}
    
    \item \textbf{Zusammenarbeit}

    \item \textbf{Standardisierung} 
    
    Dieser Punkt umfasst Anwendungen, Systeme und Prozesse. Die Verwendung von gängigen sowie standardisierten Lösungen bringt eine Reihe von Vorteilen. Sie erhöht allen voran die Nachvollziehbarkeit der Systeme und sorgt für eine Effizienzsteigerung von Prozessen \cite{Strietzel2018, Bussmann2006, Alt2017}. 
    
    Beteiligte, insbesondere die Prüfer der Bankaufsicht und die interne Revision, können so mit einem Grundwissen über die IT die meisten Systeme auf Anhieb verstehen. Die Implementierung von \emph{Sicherheitstandards \cite{IT-Grundschutz:2020, Disterer2013}} ist hierbei besonders wichtig und bietet eine Grundlage für die Einhaltung der regulatorischen Vorgaben \cite{MaRisk:2017, BAIT:2018}.
    
    Der Einsatz von Standardsoftware führt zu einem effizienteren Ablauf in ihrer Wartung und Weiterentwicklung \cite{Bussmann2006}. Die Komplexität von operativen Systemen schränkt die Einführung von Standardsoftware jedoch ein \cite[S.27]{Bussmann2006}. Das Prinzip der \emph{Standardisierung} in Verbindung mit \emph{Offenheit} ergibt die Verwendung von \enquote{Open-Source} Software, die eine höhere Effizienzsteigerung aufgrund der Wiederverwendbarkeit der Komponenten bezweckt \cite{Brockhoff2006, Gupta:2017}. 
    
    Die Einführung von Standardsoftware hat ebenfalls Auswirkungen auf eingespielte Prozesse und die daraus folgenden Einzelentscheidungen beeinflussen in ihrer Summe die Entscheidung für oder gegen den Standard \cite[Tab.1]{Manz2018}.
    
    \item \textbf{Modularisierung}
    
    \item \textbf{Nachvollziehbarkeit}
    
    \item \textbf{Automatisierung}
    
    \item \textbf{Effizienzsteigerung}
    
    Die Effizienzsteigerung wirkt sich auf die verwendeten Ressourcen und Zeitaufwand der Prozesse, Systeme oder Anwendungen aus. Dazu zählen auch Kosten- und Personalaufwand. Es stellt sich die Frage in welchen Bereichen eine Effizienzsteigerung am sinnvollsten ist.
    
    Die Effizienzsteigerung wird vor allem im Bereich der Veränderungsprozesse \enquote{Change} bezüglich ihres Zeitaufwands erforderlich. Im Bereich der betrieblichen Prozesse \enquote{Run} ist sie primär für den Aufwand der Ressourcen erforderlich.
    
    Eine Effizienzsteigerung im Betrieb ermöglicht mehr Mittel für die \enquote{Change} Prozesse \cite{Rausch2006}. Dagegen steht das Prinzip, dass Ressourcen im Gegensatz zu Zeit finanziert und eingekauft werden können. Bevor an Opportunitäten in Form einer strategischen Weiterentwicklung der Produkte \cite{Rausch2006} gedacht wird, sollte die Weiterentwicklung der hierfür vorausgesetzten Veränderungsprozesse berücksichtigt werden. Die Fähigkeit für eine kontinuierliche Anpassung \cite{Bussmann2006, Ganswindt2006} der Produkte wird mit aktuellen Methoden, wie in Abb. \ref{fig:devops} dargestellt, bereits vorausgesetzt \cite{Alt2017}.

\end{enumerate}

\section{Zusammenfassung und Ausblick}
Es ergibt sich ein Gesamtbild, dass die aufgeführten Faktoren für Veränderungsprozesse keinen Ablauf oder direkte Handlungsanweisungen implizieren. Diese leiten sich aus dem gegenwärtigen Zustand ab und beschreiben \emph{Einflüsse} für Veränderungsprozesse und haben wesentliche Auswirkungen auf ihren Ablauf. Insbesondere die antreibenden Faktoren für Veränderungsprozesse können als ihre fachlichen Anforderungen gesehen werden und die Einschränkungen als ihre Risiken. Kontinuierliche Ansätze wie \emph{DevOps} und \emph{Design-Thinking} helfen dabei die Veränderungsprozesse selbst agil und kontinuierlich anzupassen. 

Darüber hinaus wird deutlich, dass die Informatik weitreichende Kompetenzen verfügt, aus dem Zusammenhang heraus auch Probleme aus anderen Bereichen zu lösen. Beginnend von der Ausgangssituation eines vorwiegend technischen Problems haben innovationsorientiere Ansätze weitreichende Implikationen auf ihre nicht-technische Umgebung. Die Abkürzung IT sollte im Sinne einer Organisation künftig nicht für Informationstechnologie stehen sondern für \enquote{Innovation und Transformation} stehen, wodurch es Bezug auf die Eingangs- und Ausgangszustände des Paradigmas \enquote{Innovate-Design-Transform} nimmt.

Kreative Methoden sind durch die ad-hoc Generierung von Prototypen oder Beschreibungen wichtige Mittel für die Gestaltung von immateriellen Zuständen und daher für Innovationsprozesse geeignet. Die Generierung ihrer Ergebnisse ergibt sich aus der Koheränz zwischen Gegenwärtigem Zustand und zukünftigen Risiken sowie Opportunitäten. Sie bieten keine direkt implementierbare Lösungen, sind aber für eine Strategie besonders wertvoll. Die steigende Kontinuität, Geschwindigkeit und Auswirkung von Veränderungen verlangen das Treffen von schnellen Entscheidungen, wofür Agilität und Kreativität in der Strategie bedingt wird. Geschieht dies nicht kann auch kein effektiver Veränderungsprozess erzeugt werden, der Ziele aus der Strategie erfolgreich umsetzt. Mit ineffizienten Innovationsprozessen entsteht für die Transformation eine Paralyse durch Analyse, mit darauf folgender Handlungsunfähigkeit. 

Die einschränkenden Faktoren aus dem gegenwärtigen Zustand einer Organisation spielen für ihre Transformationsfähigkeit eine wesentliche Rolle. Die Durchführung der vorgenommenen Veränderungen wird dadurch erschwert. 

Für eine effektive Transformation bedingt es reibungslose Veränderungsprozesse, welche die Transformationsfähigkeit der Umgebung messen können. Danach sollte die Entwicklung von Gegenmaßnahmen als Antreibende Anlagen für die Transformationsfähigkeit priorisiert werden. Als einer der einflussreichsten Maßnahmen fallen hierunter Plattformen für die IT und Anwendungen für die Unterstützung von Steuerungs- und Überwachungsprozessen.

Das Kontinuitätsprinzip hat bereits in vielen Bereichen der Softwareentwicklung eine Grundlage für effiziente und effektive Vorgehen geschaffen. Beispiele wären \ac{CI} für die Integration von Komponenten und \ac{CD} für ein effektives und kontrolliertes Ausrollen in die Umgebung. Zusammengefasst in DevOps ergeben sie einen kontinuierlichen Lebenszyklus der Verbesserung von IT-Produkten. Das gleiche Prinzip kann auch auf das Prozessmanagement angewandt werden.

Die Digitale Transformation erzeugt viel Ungewissenheit mit darauf folgender Handlungsunfähigkeit. Das Problem an der Partizipation sind die Eintrittshürden für Organisationen mit Legacy-IT. Jeder Versuch einer Transformation wird durch die gegenseitigen Abhängigkeiten der Systeme gebremst. Organisationen beginnen die Transformation an falscher Stelle, wodurch Projekte eingestellt werden. Daher wird auch ein kontinuierlicher Lebenszyklus der Verbesserung von Veränderungsprozessen bedingt. Ein Change-Management, dass mithilfe eines starken Innovationsmanagements Anlagen für den Antrieb der Transformationsfähigkeit identifiziert und fördert und somit kontinuierlich den Innovationsmotor beschleunigt.
Das neue Paradigma \emph{innovate-design-transform} könnte mit dem zusätzlichen Punkt \emph{aggregate} ein Kontinuum formen und Grundlage für kontinuierliche Verbesserungen der Veränderungsprozesse sein. Ein erweitertes Modell mit Metriken für Innovationsfähigkeit, Gestaltungsfähigkeit, Transformationsfähigkeit

\newpage
\paragraph{Ausblick}


Die Umwandlung von sequenziellen und aufwändigen Rollouts für Changes in eine vollautomatisierte Orchestrierung mit kontinuierlicher Transformationsfähigkeit der Organisationen könnte womöglich neben der für die IT revolutionären Cloud-Plattformen eine viel größere Auswirkungen nach sich ziehen. 

Zukünftig könnten Organisationen sich mit einem kontinuierlichen \enquote{Innovate-Transform} Lebenszyklus eine unvorstellbare Schnelligkeit aufbauen.
Die Weiterentwicklung des \emph{innovate-design-transform} Paradigmas würde die Effizienz von Transformationsprozessen durch Effektivität von Innovationsprozessen mit einem entsprechend inkrementellen Antrieb steigern. Die Transformationsfähigkeit als kulturelles Prinzip könnte eine von Grund auf agile Organisation nach sich ziehen. Entsprechend würden umfassende Tätigkeiten ausgelagert oder konsolidiert werden, sodass die Kernkompetenz eines Unternehmens zukünftig aus Innovationskraft und Transformationsfähigkeit besteht. \enquote{Innovation und Transformation} verschafft der IT somit eine neue Bedeutung.